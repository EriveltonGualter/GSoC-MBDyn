\section{Inverse Dynamics Problem}
In principle, an inverse dynamics problem consists in computing the `torques'
(actuator forces) required to comply with equilibrium for a given configuration
(position, velocity and acceleration) of the entire system.

In practice, in many cases the motion of the system is not directly known
in terms of the coordinates that are used to describe it,
but rather in terms of the prescribed motion of the `joints' (the actuators)
that apply the unknown torques.
As a consequence, an inverse kinematics problem needs to be solved first,
to determine the configuration of the system up to the acceleration level
as a function of the motion prescribed to the end effector
up to the acceleration level.
Section~\ref{sec:id:facp} deals with the case of a fully actuated problem,
i.e.\ a problem whose motion is completely prescribed in terms
of joint coordinates.
Section~\ref{sec:id:uucp} deals with the case of an underactuated problem,
i.e.\ a problem whose motion is only partially prescribed in terms
of joint coordinates.

In many cases, even the motion of the actuators is not directly known.
On the contrary, the motion of another part of the system
(called the `end effector') is prescribed, and the actuators' motion
required to move the end effector as prescribed needs to be computed.
As a consequence, the inverse kinematics problem is driven
by the motion prescribed to the end effector rather than directly
to the joints.
This case is specifically dealt with in Section~\ref{sec:id:uop}.

\subsection{Nomenclature}
In the following, $\T{x}$ are the $n$ coordinates that describe
the motion of the system;
\begin{align}
	\T{\phi}\plbr{\T{x}}
	&=
	\T{0}
\end{align}
are $b$ so-called `passive' constraints, namely those that describe
the assembly of the system;
\begin{align}
	\T{\vartheta}\plbr{\T{x}}
	&=
	\T{\theta}
\end{align}
are $j$ so-called `joint coordinates', namely those that describe
the motion $\T{\theta}$ of the joints,
with $\T{\phi}_{/\T{x}} \T{\vartheta}_{/\T{x}}^+ \equiv \TT{0}$;
\begin{align}
	\T{\psi}\plbr{\T{x}}
	&=
	\T{\alpha}(t)
\end{align}
are $c$ so-called `control constraints', namely equations that prescribe
the motion of the end effector,
with $\T{\phi}_{/\T{x}} \T{\psi}_{/\T{x}}^+ \equiv \TT{0}$.
When the joint motion is directly prescribed,
$\T{\theta} \equiv \T{\alpha}(t)$
and thus
$\T{\vartheta}(\T{x}) \equiv \T{\psi}(\T{x})$.
In some cases, a partial overlap may exist between the two sets of equations.



\subsection{Fully Actuated, Collocated Problem}
\label{sec:id:facp}
In this case, $\T{\psi}(\T{x}) \equiv \T{\vartheta}(\T{x})$
and $c = j = n - b$.

\subsubsection{Inverse Kinematics: Position Subproblem}
\begin{subequations}
\begin{align}
	\T{\phi}\plbr{\T{x}}
	&=
	\T{0}
	\\
	\T{\psi}\plbr{\T{x}}
	&=
	\T{\alpha}
\end{align}
\end{subequations}
using Newton-Raphson:
\begin{align}
	\sqbr{\cvvect{
		\T{\phi}_{/\T{x}}
		\\
		\T{\psi}_{/\T{x}}
	}} \Delta\T{x}
	&=
	\cubr{\cvvect{
		-\T{\phi}\plbr{\T{x}}
		\\
		\T{\alpha}
		-
		\T{\psi}\plbr{\T{x}}
	}}
	.
\end{align}

\subsubsection{Inverse Kinematics: Velocity Subproblem}
\begin{subequations}
\begin{align}
	\T{\phi}_{/\T{x}} \dot{\T{x}}
	&=
	\T{0}
	\\
	\T{\psi}_{/\T{x}} \dot{\T{x}}
	&=
	\dot{\T{\alpha}}
\end{align}
\end{subequations}
or
\begin{align}
	\sqbr{\cvvect{
		\T{\phi}_{/\T{x}}
		\\
		\T{\psi}_{/\T{x}}
	}} \dot{\T{x}}
	&=
	\cubr{\cvvect{
		\T{0}
		\\
		\dot{\T{\alpha}}
	}}
	.
\end{align}

\subsubsection{Inverse Kinematics: Acceleration Subproblem}
\begin{subequations}
\begin{align}
	\T{\phi}_{/\T{x}} \ddot{\T{x}}
	&=
	-\plbr{\T{\phi}_{/\T{x}} \dot{\T{x}}}_{/\T{x}} \dot{\T{x}}
	\\
	\T{\psi}_{/\T{x}} \ddot{\T{x}}
	&=
	\ddot{\T{\alpha}}
	-
	\plbr{\T{\psi}_{/\T{x}} \dot{\T{x}}}_{/\T{x}} \dot{\T{x}}
\end{align}
\end{subequations}
or
\begin{align}
	\sqbr{\cvvect{
		\T{\phi}_{/\T{x}}
		\\
		\T{\psi}_{/\T{x}}
	}} \ddot{\T{x}}
	&=
	\cubr{\cvvect{
		-\plbr{\T{\phi}_{/\T{x}} \dot{\T{x}}}_{/\T{x}} \dot{\T{x}}
		\\
		\ddot{\T{\alpha}}
		-
		\plbr{\T{\psi}_{/\T{x}} \dot{\T{x}}}_{/\T{x}} \dot{\T{x}}
	}}
	.
\end{align}

\subsubsection{Inverse Dynamics Subproblem}
\begin{align}
	\TT{M} \ddot{\T{x}}
	+
	\T{\phi}_{/\T{x}}^T \T{\lambda}
	+
	\T{\psi}_{/\T{x}}^T \T{c}
	&=
	\T{f}
	\label{eq:id:facp:ids}
\end{align}
or
\begin{align}
	\sqbr{\cvvect{
		\T{\phi}_{/\T{x}}
		\\
		\T{\psi}_{/\T{x}}
	}}^T \cubr{\cvvect{
		\T{\lambda}
		\\
		\T{c}
	}}
	&=
	\T{f}
	-
	\TT{M} \ddot{\T{x}}
	\label{eq:id:facp}
	.
\end{align}
The matrix of the Newton-Raphson problem for the position
is identical to the matrices of the linear problems for the velocity
and the acceleration, while the inverse dynamics problem uses its transpose,
so the same factorization can be easily reused.



\subsection{Fully Actuated, Non-Collocated Problem}
\label{sec:id:fancp}
In this case, $\T{\psi}(\T{x}) \neq \T{\vartheta}(\T{x})$,
but still $c = j = n - b$.

\subsubsection{Inverse Kinematics: Position, Velocity and Acceleration Subproblems}
The position, velocity and acceleration subproblems
that define the inverse kinematics problem are identical
to those of the collocated case of Section~\ref{sec:id:facp}.

\subsubsection{Inverse Dynamics Subproblem}
\begin{align}
	\TT{M} \ddot{\T{x}}
	+
	\T{\phi}_{/\T{x}}^T \T{\lambda}
	+
	\T{\vartheta}_{/\T{x}}^T \T{c}
	&=
	\T{f}
	\label{eq:id:fancp:ids}
\end{align}
or
\begin{align}
	\sqbr{\cvvect{
		\T{\phi}_{/\T{x}}
		\\
		\T{\vartheta}_{/\T{x}}
	}}^T \cubr{\cvvect{
		\T{\lambda}
		\\
		\T{c}
	}}
	&=
	\T{f}
	-
	\TT{M} \ddot{\T{x}}
	\label{eq:id:fancp}
	.
\end{align}
The inverse dynamics problem uses a different matrix from the transpose
of that that was used for the inverse kinematics subproblems.



\subsection{Underdetermined, Underactuated but Collocated Problem}
\label{sec:id:uucp}
In this case, again $\T{\psi}(\T{x}) \equiv \T{\vartheta}(\T{x})$
but $c = j < n - b$.

\subsubsection{Inverse Kinematics: Position Subproblem}
Constraint equations:
\begin{subequations}
\begin{align}
	\T{\phi}\plbr{\T{x}}
	&=
	\T{0}
	\\
	\T{\psi}\plbr{\T{x}}
	&=
	\T{\alpha}(t)
\end{align}
\end{subequations}
with $\T{\phi}_{/\T{x}}$ and $\T{\psi}_{/\T{x}}$ rectangular, full row rank.
The problem is underdetermined;
as a consequence, some criteria are needed to find an optimal solution.

The problem can be augmented by
\begin{align}
	\TT{K} \plbr{\T{x} - \T{x}_0}
	+
	\T{\phi}_{/\T{x}}^T \T{\lambda}
	+
	\T{\psi}_{/\T{x}}^T \T{\mu}
	&=
	\T{0}
\end{align}
This nonlinear problem is solved for $\T{x}$, $\T{\lambda}$ and $\T{\mu}$
to convergence ($\TT{K}$, $\T{\phi}_{/\T{x}}$ and $\T{\psi}_{/\T{x}}$
may further depend on $\T{x}$).
It corresponds to a least-squares solution for $\T{x} - \T{x}_0$,
where the quadratic form
\begin{align}
	J_{\T{x}}
	&=
	\frac{1}{2} \plbr{\T{x} - \T{x}_0}^T \TT{K} \plbr{\T{x} - \T{x}_0}
\end{align}
is minimized subjected to $\T{\phi} = \T{0}$ and $\T{\psi} = \T{\alpha}$,
and weighted by matrix $\T{K}$;
$\T{x}_0$ is a reference solution.
Matrix $\TT{K}$ can further depend on $\T{x}$.

The reference solution $\T{x}_0$ can be used to further control
the quality of the solution.
For example, it may represent a prescribed tentative,
although possibly incompatible, trajectory.
Another option consists in augmenting $J_{\T{x}}$
with another quadratic form
\begin{align}
	J_{\T{x}}
	&=
	\frac{1}{2} \plbr{\T{x} - \T{x}_0}^T \TT{K} \plbr{\T{x} - \T{x}_0}
	+
	w_{\T{x}} \frac{1}{2} \plbr{\T{x} - \T{x}_\text{prev}}^T \TT{M} \plbr{\T{x} - \T{x}_\text{prev}}
\end{align}
where $\T{x}_\text{prev}$ is the value of $\T{x}$ at the previous time step.
This modified quadratic form weights the rate of change of $\T{x}$ within
two consecutive steps.
As a consequence, minimal position changes (i.e.\ velocities),
weighted by the mass of the system, are sought.
The corresponding problem is
\begin{subequations}
\begin{align}
	\TT{K} \plbr{\T{x} - \T{x}_0}
	+
	w_{\T{x}} \TT{M} \plbr{\T{x} - \T{x}_\text{prev}}
	+
	\T{\phi}_{/\T{x}}^T \T{\lambda}
	+
	\T{\psi}_{/\T{x}}^T \T{\mu}
	&=
	\T{0}
	\\
	\T{\phi}\plbr{\T{x}}
	&=
	\T{0}
	\\
	\T{\psi}\plbr{\T{x}}
	&=
	\T{\alpha}
	,
\end{align}
\end{subequations}
or
\begin{subequations}
\begin{align}
	\plbr{\TT{K} + w_{\T{x}} \TT{M}} \T{x}
	+
	\T{\phi}_{/\T{x}}^T \T{\lambda}
	+
	\T{\psi}_{/\T{x}}^T \T{\mu}
	&=
	\TT{K} \T{x}_0 + w_{\T{x}} \TT{M} \T{x}_\text{prev}
	\\
	\T{\phi}\plbr{\T{x}}
	&=
	\T{0}
	\\
	\T{\psi}\plbr{\T{x}}
	&=
	\T{\alpha}
	.
\end{align}
\end{subequations}
Using Newton-Raphson:
\begin{align}
	\sqbr{\matr{ccc}{
		\TT{K} + w_{\T{x}} \TT{M} & \T{\phi}_{/\T{x}}^T & \T{\psi}_{/\T{x}}^T
		\\
		\T{\phi}_{/\T{x}} & \TT{0} & \TT{0}
		\\
		\T{\psi}_{/\T{x}} & \TT{0} & \TT{0}
	}} \cubr{\cvvect{
		\Delta\T{x}
		\\
		\Delta\T{\lambda}
		\\
		\Delta\T{\mu}
	}}
	&=
	\cubr{\cvvect{
		\TT{K} \plbr{\T{x}_0 - \T{x}}
		+
		w_{\T{x}} \TT{M} \plbr{\T{x}_\text{prev} - \T{x}}
		-
		\T{\phi}_{/\T{x}}^T \T{\lambda}
		-
		\T{\psi}_{/\T{x}}^T \T{\mu}
		\\
		-\T{\phi}\plbr{\T{x}}
		\\
		\T{\alpha}(t)
		-
		\T{\psi}\plbr{\T{x}}
	}}
\end{align}

\subsubsection{Inverse Kinematics: Velocity Subproblem}
Constraint first derivative:
\begin{subequations}
\begin{align}
	\TT{R} \plbr{\dot{\T{x}} - \dot{\T{x}}_0}
	+
	\T{\phi}_{/\T{x}}^T \T{\lambda}
	+
	\T{\psi}_{/\T{x}}^T \T{\mu}
	&=
	\T{0}
	\\
	\T{\phi}_{/\T{x}} \dot{\T{x}}
	&=
	\T{0}
	\\
	\T{\psi}_{/\T{x}} \dot{\T{x}}
	&=
	\dot{\T{\alpha}}
	.
\end{align}
\end{subequations}
In analogy with the position constraint case,
it corresponds to minimizing the quadratic form
\begin{align}
	J_{\dot{\T{x}}}
	&=
	\frac{1}{2} \plbr{\dot{\T{x}} - \dot{\T{x}}_0}^T \TT{R} \plbr{\dot{\T{x}} - \dot{\T{x}}_0}
	,
\end{align}
which can be augmented as well, resulting in
\begin{align}
	J_{\dot{\T{x}}}
	&=
	\frac{1}{2} \plbr{\dot{\T{x}} - \dot{\T{x}}_0}^T \TT{R} \plbr{\dot{\T{x}} - \dot{\T{x}}_0}
	+
	w_{\dot{\T{x}}} \frac{1}{2} \plbr{\dot{\T{x}} - \dot{\T{x}}_\text{prev}}^T \TT{M} \plbr{\dot{\T{x}} - \dot{\T{x}}_\text{prev}}
	,
\end{align}
to minimize the velocity increment between two consecutive time steps.
The corresponding problem is
\begin{subequations}
\begin{align}
	\plbr{\TT{R} + w_{\dot{\T{x}}} \TT{M}} \dot{\T{x}}
	+
	\T{\phi}_{/\T{x}}^T \T{\lambda}
	+
	\T{\psi}_{/\T{x}}^T \T{\mu}
	&=
	\TT{R} \dot{\T{x}}_0
	+
	w_{\dot{\T{x}}} \TT{M} \dot{\T{x}}_\text{prev}
	\\
	\T{\phi}_{/\T{x}} \dot{\T{x}}
	&=
	\T{0}
	\\
	\T{\psi}_{/\T{x}} \dot{\T{x}}
	&=
	\dot{\T{\alpha}}
	,
\end{align}
\end{subequations}
or
\begin{align}
	\sqbr{\matr{ccc}{
		\TT{R} + w_{\dot{\T{x}}} \TT{M} & \T{\phi}_{/\T{x}}^T & \T{\psi}_{/\T{x}}^T
		\\
		\T{\phi}_{/\T{x}} & \TT{0} & \TT{0}
		\\
		\T{\psi}_{/\T{x}} & \TT{0} & \TT{0}
	}} \cubr{\cvvect{
		\dot{\T{x}}
		\\
		\T{\lambda}
		\\
		\T{\mu}
	}}
	&=
	\cubr{\cvvect{
		\TT{R} \dot{\T{x}}_0 + w_{\dot{\T{x}}} \TT{M} \dot{\T{x}}_\text{prev}
		\\
		\T{0}
		\\
		\dot{\T{\alpha}}
	}}
	.
\end{align}
The constraint derivative problem is linear in $\dot{\T{x}}$;
the same symbols $\T{\lambda}$ and $\T{\mu}$ are used for the multipliers
since they are ineffective (their value is never used).
The problems can be solved sequentially.
Only in case $\TT{K} \equiv \TT{R}$ and $w_{\T{x}} \equiv w_{\dot{\T{x}}}$
both problems use the same matrix and thus the same factorization
can be reused.

\subsubsection{Inverse Dynamics Subproblem}
The second derivative of the constraint cannot be resolved
as in the fully actuated case, otherwise it could yield accelerations
that cannot be imposed by the constraints.
On the contrary, the inverse dynamics problem of Eq.~(\ref{eq:id:facp:ids})
is directly solved, yielding both the accelerations and the torques,
\begin{subequations}
\begin{align}
	\TT{M} \ddot{\T{x}}
	+
	\T{\phi}_{/\T{x}} \T{\lambda}
	+
	\T{\psi}_{/\T{x}} \T{c}
	&=
	\T{f}
	\\
	\T{\phi}_{/\T{x}} \ddot{\T{x}}
	&=
	-\plbr{\T{\phi}_{/\T{x}} \dot{\T{x}}}_{/\T{x}} \dot{\T{x}}
	\\
	\T{\psi}_{/\T{x}} \ddot{\T{x}}
	&=
	\ddot{\T{\alpha}}
	-
	\plbr{\T{\psi}_{/\T{x}} \dot{\T{x}}}_{/\T{x}} \dot{\T{x}}
\end{align}
\end{subequations}
or
\begin{align}
	\sqbr{\matr{ccc}{
		\TT{M} & \T{\phi}_{/\T{x}}^T & \T{\psi}_{/\T{x}}^T
		\\
		\T{\phi}_{/\T{x}} & \TT{0} & \TT{0}
		\\
		\T{\psi}_{/\T{x}} & \TT{0} & \TT{0}
	}} \cubr{\cvvect{
		\ddot{\T{x}}
		\\
		\T{\lambda}
		\\
		\T{c}
	}}
	&=
	\cubr{\cvvect{
		\T{f}
		\\
		-\plbr{\T{\phi}_{/\T{x}} \dot{\T{x}}}_{/\T{x}} \dot{\T{x}}
		\\
		\ddot{\T{\alpha}}
		-
		\plbr{\T{\psi}_{/\T{x}} \dot{\T{x}}}_{/\T{x}} \dot{\T{x}}
	}}
	\label{eq:id:uucp:idsp}
\end{align}
As a consequence, the same structure of the inverse kinematic problems
is achieved, and the last problem of Eq.~(\ref{eq:id:uucp:idsp})
directly yields both accelerations and multipliers.




\subsection{Underdetermined, Overcontrolled Problem}
\label{sec:id:uop}
In this case, $\T{\psi}(\T{x}) \neq \T{\vartheta}(\T{x})$,
$c < n - b$, $j = n - b$, which implies $c < j$.
This is the case, for example, of an inverse biomechanics problem,
where each degree of freedom is a joint commanded by a set of muscles.
The inverse kinematics problem is solved by prescribing the motion
of some part (e.g.\ a hand or a foot) in order to determine an `optimal'
(e.g.\ in terms of maximal ergonomy) motion.
Then, an inverse dynamics problem is computed by freeing the part
whose motion was initially prescribed, and by computing the torque
required by all joints.

\subsubsection{Inverse Kinematics: Position and Velocity Subproblems}
The constraint and its first derivative are dealt with
as in Section~\ref{sec:id:uucp}.

\subsubsection{Inverse Kinematics: Acceleration Subproblem}
Constraint second derivative:
\begin{subequations}
\begin{align}
	\TT{M} \plbr{\ddot{\T{x}} - \ddot{\T{x}}_0}
	+
	\T{\phi}_{/\T{x}}^T \T{\lambda}
	+
	\T{\psi}_{/\T{x}}^T \T{\mu}
	&=
	\T{0}
	\\
	\T{\phi}_{/\T{x}} \ddot{\T{x}}
	&=
	-\plbr{\T{\phi}_{/\T{x}} \dot{\T{x}}}_{/\T{x}} \dot{\T{x}}
	\\
	\T{\psi}_{/\T{x}} \ddot{\T{x}}
	&=
	\ddot{\T{\alpha}}
	-
	\plbr{\T{\psi}_{/\T{x}} \dot{\T{x}}}_{/\T{x}} \dot{\T{x}}
	.
\end{align}
\end{subequations}
In analogy with the position constraint case,
it corresponds to minimizing the quadratic form
\begin{align}
	J_{\ddot{\T{x}}}
	&=
	\frac{1}{2} \plbr{\ddot{\T{x}} - \ddot{\T{x}}_0}^T \TT{M} \plbr{\ddot{\T{x}} - \ddot{\T{x}}_0}
	,
\end{align}
which can be augmented as well, resulting in
\begin{align}
	J_{\ddot{\T{x}}}
	&=
	\frac{1}{2} \plbr{\ddot{\T{x}} - \ddot{\T{x}}_0}^T \TT{M} \plbr{\ddot{\T{x}} - \ddot{\T{x}}_0}
	+
	w_{\ddot{\T{x}}} \frac{1}{2} \plbr{\ddot{\T{x}} - \ddot{\T{x}}_\text{prev}}^T \TT{M} \plbr{\ddot{\T{x}} - \ddot{\T{x}}_\text{prev}}
	,
\end{align}
to minimize the acceleration increment between two consecutive time steps.
The corresponding problem is
\begin{subequations}
\begin{align}
	\plbr{1 + w_{\dot{\T{x}}}} \TT{M} \ddot{\T{x}}
	+
	\T{\phi}_{/\T{x}}^T \T{\lambda}
	+
	\T{\psi}_{/\T{x}}^T \T{\mu}
	&=
	\TT{M} \ddot{\T{x}}_0
	+
	w_{\ddot{\T{x}}} \TT{M} \ddot{\T{x}}_\text{prev}
	\\
	\T{\phi}_{/\T{x}} \ddot{\T{x}}
	&=
	-\plbr{\T{\phi}_{/\T{x}} \dot{\T{x}}}_{/\T{x}} \dot{\T{x}}
	\\
	\T{\psi}_{/\T{x}} \ddot{\T{x}}
	&=
	\ddot{\T{\alpha}}
	-
	\plbr{\T{\psi}_{/\T{x}} \dot{\T{x}}}_{/\T{x}} \dot{\T{x}}
	,
\end{align}
\end{subequations}
or
\begin{align}
	\sqbr{\matr{ccc}{
		\plbr{1 + w_{\ddot{\T{x}}}} \TT{M} & \T{\phi}_{/\T{x}}^T & \T{\psi}_{/\T{x}}^T
		\\
		\T{\phi}_{/\T{x}} & \TT{0} & \TT{0}
		\\
		\T{\psi}_{/\T{x}} & \TT{0} & \TT{0}
	}} \cubr{\cvvect{
		\ddot{\T{x}}
		\\
		\T{\lambda}
		\\
		\T{\mu}
	}}
	&=
	\cubr{\cvvect{
		\TT{M} \ddot{\T{x}}_0 + w_{\ddot{\T{x}}} \TT{M} \ddot{\T{x}}_\text{prev}
		\\
		-\plbr{\T{\phi}_{/\T{x}} \dot{\T{x}}}_{/\T{x}} \dot{\T{x}}
		\\
		\ddot{\T{\alpha}}
		-
		\plbr{\T{\psi}_{/\T{x}} \dot{\T{x}}}_{/\T{x}} \dot{\T{x}}
	}}
	.
\end{align}

\subsubsection{Inverse Dynamics Subproblem}
The inverse dynamics subproblem is identical to that
of the fully determined, non-collocated case, Eq.~(\ref{eq:id:fancp:ids}).



\subsubsection{Implementation}
\begin{itemize}
\item during `position' inverse kinematics substep:
\begin{itemize}
\item assemble $\T{\phi}_{/\T{x}}$ and $\T{\phi}_{/\T{x}}^T$
for passive constraints
\item assemble $\T{\psi}_{/\T{x}}$ and $\T{\psi}_{/\T{x}}^T$
for prescribed motion
\item assemble equations related to $\T{\vartheta}_{/\T{x}}$
as $\T{c} = \T{0}$ to neutralize them
\item assemble dummy springs
\item optionally assemble mass matrix contribution weighted by $w_{/\T{x}}$
\end{itemize}


\item during `velocity' inverse kinematics substep:
\begin{itemize}
\item assemble $\T{\phi}_{/\T{x}}$ and $\T{\phi}_{/\T{x}}^T$
for passive constraints
\item assemble $\T{\psi}_{/\T{x}}$ and $\T{\psi}_{/\T{x}}^T$
for prescribed motion
\item assemble equations related to $\T{\vartheta}_{/\T{x}}$
as $\T{c} = \T{0}$ to neutralize them
\item assemble dummy dampers
\item optionally assemble mass matrix contribution weighted by $w_{/\dot{\T{x}}}$
\end{itemize}


\item during `acceleration' inverse kinematics substep:
\begin{itemize}
\item assemble $\T{\phi}_{/\T{x}}$ and $\T{\phi}_{/\T{x}}^T$
for passive constraints
\item assemble $\T{\psi}_{/\T{x}}$ and $\T{\psi}_{/\T{x}}^T$
for prescribed motion
\item assemble equations related to $\T{\vartheta}_{/\T{x}}$
as $\T{c} = \T{0}$ to neutralize them
\item assemble mass matrix contribution, optionally weighted by $(1 + w_{/\ddot{\T{x}}})$
\end{itemize}


\item during inverse dynamics substep:
\begin{itemize}
\item assemble $\T{\phi}_{/\T{x}}$ and $\T{\phi}_{/\T{x}}^T$
for passive constraints
\item assemble $\T{\vartheta}_{/\T{x}}$ and $\T{\vartheta}_{/\T{x}}^T$
for torques
\item assemble equations related to $\T{\psi}_{/\T{x}}$
as $\T{\mu} = \T{0}$ to neutralize them
\item assemble other elements (external forces, springs, etc.)
\end{itemize}
\end{itemize}
